\documentclass[]{vvsu}
\usepackage{csquotes}

\vvsuyear{2025}

%%%%%%%%%%%%%%%%%%%

\usepackage{graphicx} % для изображений
\usepackage{tabularray} % для таблиц
\usepackage{siunitx} % для обозначений (процент, градус)
\usepackage{listings} % для листингов кода

% Список путей, где будут искаться изображения и файлы
\graphicspath{{images/}}

% Файл со списком источников (не используется)
% \addbibresource{./references.bib}

% Автор документа
\author{А.М. Сахарюк}

% Настройка стилей для листингов кода
\input{listing_styles.tex}

%%%%%%%%%%%%%%%%%%%

\begin{document}

% Шапка
\vvsuhead{\linespread{1}\selectfont{}МИНОБРНАУКИ РОССИИ\\
\vspace{10pt}Федеральное государственное бюджетное образовательное учреждение\\
высшего образования\\
\fontsize{13}{13}\selectfont{}<<ВЛАДИВОСТОКСКИЙ ГОСУДАРСТВЕННЫЙ УНИВЕРСИТЕТ>>\\
(ФГБОУ ВО <<ВВГУ>>)\\
\vspace{10pt}\fontsize{12}{12}\selectfont{}ИНСТИТУТ ИНФОРМАЦИОННЫХ ТЕХНОЛОГИЙ И АНАЛИЗА ДАННЫХ\\
КАФЕДРА ИНФОРМАЦИОННЫХ ТЕХНОЛОГИЙ И СИСТЕМ}

% Название отчета
\title{Отчет\\по лабораторной работе №5}
\subtitle{по дисциплине\\<<Информатика и программирование>>}

% Участники работы
\member{Студент\\ гр. БИН-25-2}{А.М. Сахарюк}
\member{Ассистент\\ преподавателя}{М.В. Водяницкий}

% Вывод титульника
\maketitle

% Задание
% Задание
\begin{addition}{Задание}
  Выполнить задания на Python и оформить отчет по стандартам ВВГУ.

  \textit{\textbf{Задание 1.}}  
  Дан список из 10 различных целых чисел. Необходимо найти в нем число 3 и заменить на 30.

  \textit{\textbf{Задание 2.}}  
  Дан список из 5 целых чисел. Необходимо превратить его в список квадратов этих чисел.

  \textit{\textbf{Задание 3.}}  
  Имеется список различных целых чисел. Программа должна найти наибольшее из чисел списка и разделить его на длину списка.
  
  \textit{\textbf{Задание 4.}}  
  Имеется кортеж из нескольких произвольных элементов. Необходимо этот кортеж отсортировать. Если хотя бы один элемент не является числом, то кортеж остается неизменным.

  \textit{\textbf{Задание 5.}}  
  Имеется словарь товаров в магазине. Необходимо найти товар с минимальной и максимальной ценой.

  \textit{\textbf{Задание 6.}}  
  Имеется список произвольных элементов. Необходимо на основе этого списка создать словарь, где каждый элемент списка будет и ключом, и значением.

  \textit{\textbf{Задание 7.}}  
  Имеется словарь перевода английских слов на русский, где ключ английского слово, значение - русского. Необходимо реализовать программу которая получает на ввод русское слово и результатом выдает перевод на английский.

  \textit{\textbf{Задание 8.}}  
  Реализовать игру Камень-Ножницы-Бумага-Ящерица-Спок. Программа должна запрашивать у пользователя ввод одного из вариантов. Второй вариант случайно генерирует сама программа и возвращает победителя.
  Правила игры следующие:
  \begin{vvsu_list}
    \item Ножницы режут бумагу
    \item Бумага покрывает камень
    \item Ящерица отравляет Спока
    \item Спок ломает ножницы 
    \item Ножницы обезглавливают ящерицу
    \item Ящерица съедает бумагу
    \item Бумага подставляет Спока
    \item Спок испаряет камень
    \item Камень разбивает ножницы
  \end{vvsu_list}

  \textit{\textbf{Задание 9.}}  
  Дан список слов - например:

  ["яблоко", "груша", "банан", "киви", "апельсин", "ананас"]
  Необходимо создать новый словарь, где:
  \begin{vvsu_list}
    \item Ключом будет первая буква слова
    \item Значением - список всех слов, начинающихся с этой буквы
  \end{vvsu_list}
  Пример результата:

  \{'я': ['яблоко'], 'г': ['груша'], 'б': ['банан'], 'к': ['киви'], 'а': ['апельсин', 'ананас']\}

  \textit{\textbf{Задание 10.}}  
  Дан список кортежей, где каждый кортеж содержит имя студента и его оценки, например:
  [("Анна", [5, 4, 5]), ("Иван", [3, 4, 4]), ("Мария", [5, 5, 5])]
  Необходимо:
  \begin{vvsu_list}
    \item Создать словарь, где ключ - имя студента, значение - его средняя оценка
    \item Найти студента с наибольшей средней оценкой и вывести его имя \newline и средний балл
  \end{vvsu_list}
  Пример результата:
  Мария имеет наивысший средний балл: 5.0
\end{addition}

% Содержание
\toc

% Глава - Выполнение работы
\section{Выполнение работы}

% Подглава - Задание 1
\subsection{Задание 1}

В данном задании был объявлен список содержащий 10 элементов и заполненый случайными числами в диапазоне от 0 до 10.
Программа ищет число 3 в списке и заменяет  его на 30 
На рисунке \ref{fig:code_task_1} представлен код полученной программы.

\begin{vvsu_figure}{Листинг программы для задания 1}{fig:code_task_1}
  \begin{minipage}{.75\textwidth}
    \lstinputlisting[language=Python,basicstyle=\fontsize{10}{10}\linespread{1}\selectfont\ttfamily]{code/task1.py}
  \end{minipage}
\end{vvsu_figure}

Пояснение работы программы:
\begin{vvsu_list}
  \item импортируется стандартная библиотека random
  \item Объявляется список list\_of\_numbers
  \item В список добавляется 10 случайных чисел в диапазоне от  0 до 10
  \item Через цикл перебираются значения списка
  \item Eсли обнаруживается число 3, то оно заменяется на 30
  \item Список выводится  на экран
\end{vvsu_list}

После выполнения программы в консоль выводится список после выполнения  всех операций над ним.

% Подглава - Задание 2
\subsection{Задание 2}

В данном задании был объявлен список содержащий 5 элементов и заполненый случайными числами в диапазоне от 0 до 100.
Программа заменяет все числа в списке на их квадраты.
На рисунке \ref{fig:code_task_2} представлен код полученной программы.

\begin{vvsu_figure}{Листинг программы для задания 2}{fig:code_task_2}
  \begin{minipage}{.75\textwidth}
    \lstinputlisting[language=Python,basicstyle=\fontsize{10}{10}\linespread{1}\selectfont\ttfamily]{code/task2.py}
  \end{minipage}
\end{vvsu_figure}

Пояснение работы программ\begin{vvsu_figure}{Листинг программы для задания 5}{fig:code_task_5}
  \begin{minipage}{.75\textwidth}
    \lstinputlisting[language=Python,basicstyle=\fontsize{10}{10}\linespread{1}\selectfont\ttfamily]{code/task5.py}
  \end{minipage}
\end{vvsu_figure}ы:
\begin{vvsu_list}
  \item импортируется стандартная библиотека random
  \item Объявляется список list\_of\_numbers
  \item В список добавляется 5 случайных чисел в диапазоне от  0 до 100
  \item Через цикл перебираются значения списка
  \item каждое число возводится в квадрат и возвращается в список
  \item Список выводится  на экран
\end{vvsu_list}

В результате выполнения программы, в консоль выводится список после выполнения  всех операций над ним.

% Подглава - Задание 3ыы
\subsection{Задание 3}

В данном задании был объявлен список содержащий случайное количество элементов от 1 до 1000 и заполненый случайными числами в диапазоне от 0 до 10000.
Программа находит самое большое  число и делит его на длину списка
На рисунке \ref{fig:code_task_3} представлен код полученной программы.

\begin{vvsu_figure}{Листинг программы для задания 3}{fig:code_task_3}
  \begin{minipage}{.75\textwidth}
    \lstinputlisting[language=Python,basicstyle=\fontsize{10}{10}\linespread{1}\selectfont\ttfamily]{code/task3.py}
  \end{minipage}
\end{vvsu_figure}

Пояснение работы программы:
\begin{vvsu_list}
  \item импортируется стандартная библиотека random
  \item Объявляется список list\_of\_numbers
  \item В список добавляется от 1 до 1000 случайных чисел в диапазоне от  0 до 10000
  \item Через функцию len() находится длина списска
  \item Через функцию max() находится наибольшее число в списске
  \item На экран выводится максимальное число в списске делёное на длину списска.
\end{vvsu_list}

В результате выполнения программы, в консоль выводится результат выполнения формулы.

% Подглава - Задание 4
\subsection{Задание 4}

В данном задании от пользователя требуется вести что либо с клавиатуры через пробел. Программа создаёт корте, который пытается отсортировать. Если в кортеже находится не число, то кортеж не изменяется.
На рисунке \ref{fig:code_task_4} представлен код полученной программы.

\begin{vvsu_figure}{Листинг программы для задания 4}{fig:code_task_4}
  \begin{minipage}{.75\textwidth}
    \lstinputlisting[language=Python,basicstyle=\fontsize{10}{10}\linespread{1}\selectfont\ttfamily]{code/task4.py}
  \end{minipage}
\end{vvsu_figure}

Пояснение работы программы:
\begin{vvsu_list}
  \item У пользователя запрашивается ввод
  \item Ввод пользователя записывается в переменную user\_input.
  \item Объявляется user\_tuple равный разделённому по пробелам вводу пользователя
  \item Каждое значение в кортеже пытается преобразоваться в целочисленное значение
  \item В случае удачи - кортеж сортируется, в случае неудачи - кортеж не сортируется
  \item Кортеж выводится  на экран
\end{vvsu_list}

В результате выполнения программы, в консоль выводится список после выполнения  всех операций над ним.

% Подглава - Задание 5
\subsection{Задание 5}

В данном задании дан словарь наименований товаровв и их цен.
Программа должна найти товар с максимальной ценой и вывести его на экран.
На рисунке \ref{fig:code_task_5} представлен код полученной программы.

\begin{vvsu_figure}{Листинг программы для задания 5}{fig:code_task_5}
  \begin{minipage}{.75\textwidth}
    \lstinputlisting[language=Python,basicstyle=\fontsize{10}{10}\linespread{1}\selectfont\ttfamily]{code/task5.py}
  \end{minipage}
\end{vvsu_figure}

Пояснение работы программы:
\begin{vvsu_list}
  \item задаётся словарь goods со значениями "яблоко": 100, "банан": 80, "груша": 120
  \item Объявляется переменная max\_price с типом string
  \item Через цикл перебираются значения словаря
  \item каждое значение проверяется на условие: если max\_price пуста, или ценна товара больше чем в max\_price, то в max\_price записывается данное значениие
  \item а экран выводится максимальная цена товара
\end{vvsu_list}

В результате выполнения программы в консоль выводится максимальная цена товара данного списска.

% Подглава - Задание 6
\subsection{Задание 6}

В данном задании от пользователя требуется ввести любые значения через пробел.
Программа должна для каждого значения введённого пользователем создать вход в словарь, где значение является и ключём  и значением.
На рисунке \ref{fig:code_task_6} представлен код программы.

\begin{vvsu_figure}{Листинг программы для задания 6}{fig:code_task_6}
  \begin{minipage}{.75\textwidth}
    \lstinputlisting[language=Python,basicstyle=\fontsize{10}{10}\linespread{1}\selectfont\ttfamily]{code/task6.py}
  \end{minipage}
\end{vvsu_figure}

Пояснение работы программы:
\begin{vvsu_list}
  \item У пользователя запрашивается ввод
  \item Ввод пользователя записывается в переменную user\_input.
  \item Объявляется user\_list равный разделённому по пробелам вводу пользователя
  \item Объявляется user\_dict
  \item Через цикл перебираются значения ввода пользователя
  \item В словарь вводится значение где ключ и значение - одно и то же
  \item В консоль выводится словарь после операций
\end{vvsu_list}

В результате выполнения программы в консоль выводится  словарь после операций.

% Подглава - Задание 7
\subsection{Задание 7}

В данном задании программе требуется вывести перевод слова из словаря.
На рисунке \ref{fig:code_task_7} представлен код программы.

\begin{vvsu_figure}{Листинг программы для задания 7}{fig:code_task_7}
  \begin{minipage}{.75\textwidth}
    \lstinputlisting[language=Python,basicstyle=\fontsize{10}{10}\linespread{1}\selectfont\ttfamily]{code/task7.py}
  \end{minipage}
\end{vvsu_figure}

Пояснение работы программы:
\begin{vvsu_list}
  \item У пользователя запрашиввается  ввод
  \item Через цикл перебирается словарь для перевод
  \item Если ввод пользователя содержится в элементе словарая, то его перевод выводится на экран
\end{vvsu_list}

В результате выполнения программы, в консоль выводится перевод введёного пользователем слова.

% Подглава - Задание 8
\subsection{Задание 8}

В данном задании нужно написать игру Камень-Ножницы-Бумага-Ящерица-Спок.
От пользователя требуется выбрать свой ход. Программа выводит выбор бота и то, кто победил.
На рисунке \ref{fig:code_task_8} представлен код программы.

\begin{vvsu_figure}{Листинг программы для задания 8}{fig:code_task_8}
  \begin{minipage}{.75\textwidth}
    \lstinputlisting[language=Python,basicstyle=\fontsize{10}{10}\linespread{1}\selectfont\ttfamily]{code/task8.py}
  \end{minipage}
\end{vvsu_figure}

Пояснение работы программы:
\begin{vvsu_list}
  \item ЗЗадаётся словарь Со следующими значениями: Выбор хода, список - 1 противоположный ход, сообщение о победе, 2 противоположный ход, сообщение о победе.
  \item От пользователя запрашивается ввод.
  \item Задаётся случайный выбор бота
  \item Проверяется кто победил и выводиться на экран сообщение о победе.
\end{vvsu_list}

В результате выполнения программы, по правилам игры Камень-Ножницы-Бумага-Ящерица-Спок, у пользователя запрашивается ввод, и генерируется резкльтат, кто победил.

% Подглава - Задание 9
\subsection{Задание 9}

В данном задании требуется получить списсок элементов, найти первые буквы элементы и вывести на экран все элементы списка начинающиеся с этой буквы.
На рисунке \ref{fig:code_task_9} представлен код программы.

\begin{vvsu_figure}{Листинг программы для задания 9}{fig:code_task_9}
  \begin{minipage}{.75\textwidth}
    \lstinputlisting[language=Python,basicstyle=\fontsize{10}{10}\linespread{1}\selectfont\ttfamily]{code/task9.py}
  \end{minipage}
\end{vvsu_figure}

Пояснение работы программы:
\begin{vvsu_list}
  \item Задаётся список слов.
  \item Задаётся словарь result.
  \item Каждый элемент списска перебирается, и в случае обнаружения новой буквы, она добавляется, как ключ в словарь.
  \item Если обнаруживается слово на то же букву, то оно добавляется в списсок по ключу словаря.
  \item На экран выводиться готовый словарь.
\end{vvsu_list}

В результате выполнения программы в консоль выводится готовый словарь.

% Подглава - Задание 10
\subsection{Задание 10}

В данном задании требуется получить словарь с именами студентов и списском их оценок. Требуется перебрать список и найти средний балл студентов, затем вывести студента с высшим баллом.
На рисунке \ref{fig:code_task_10} представлен код программы.

\begin{vvsu_figure}{Листинг программы для задания 10}{fig:code_task_10}
  \begin{minipage}{.75\textwidth}
    \lstinputlisting[language=Python,basicstyle=\fontsize{10}{10}\linespread{1}\selectfont\ttfamily]{code/task10.py}
  \end{minipage}
\end{vvsu_figure}

Пояснение работы программы:
\begin{vvsu_list}
  \item Задаётся кортеж students, С именами студентов и списском их оценок.
  \item Задаётся словарь average\_grades.
  \item Каждый элемент кортежа перебирается и записыватся в словарь в порядке: имя студента, средняя оценка.
  \item Вычисляется студент с наивысшим  средним баллом, и \break записывается в переменную
  \item На экран выводится имя лучшего студента и его средний балл.
\end{vvsu_list}

В результате выполнения данной программы выводится в консоль имя лучшего студента и его средний балл.

% \begin{application}{Приложение А}{Приложение А\\ \vspace{1em} Описание процесса работы с драйверами}
%   Самое обычное тестовое приложение
% \end{application}

\end{document}
