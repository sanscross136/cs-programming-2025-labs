\documentclass[]{vvsu}
\usepackage{csquotes}

\vvsuyear{2025}

%%%%%%%%%%%%%%%%%%%

\usepackage{graphicx} % для изображений
\usepackage{tabularray} % для таблиц
\usepackage{siunitx} % для обозначений (процент, градус)
\usepackage{listings} % для листингов кода

% Список путей, где будут искаться изображения и файлы
\graphicspath{{images/}}

% Файл со списком источников (не используется)
% \addbibresource{./references.bib}

% Автор документа
\author{А.М. Сахарюк}

% Настройка стилей для листингов кода
\input{listing_styles.tex}

%%%%%%%%%%%%%%%%%%%

\begin{document}

% Шапка
\vvsuhead{\linespread{1}\selectfont{}МИНОБРНАУКИ РОССИИ\\
\vspace{10pt}Федеральное государственное бюджетное образовательное учреждение\\
высшего образования\\
\fontsize{13}{13}\selectfont{}<<ВЛАДИВОСТОКСКИЙ ГОСУДАРСТВЕННЫЙ УНИВЕРСИТЕТ>>\\
(ФГБОУ ВО <<ВВГУ>>)\\
\vspace{10pt}\fontsize{12}{12}\selectfont{}ИНСТИТУТ ИНФОРМАЦИОННЫХ ТЕХНОЛОГИЙ И АНАЛИЗА ДАННЫХ\\
КАФЕДРА ИНФОРМАЦИОННЫХ ТЕХНОЛОГИЙ И СИСТЕМ}

% Название отчета
\title{Отчет\\по лабораторной работе №4}
\subtitle{по дисциплине\\<<Информатика и программирование>>}

% Участники работы
\member{Студент\\ гр. БИН-25-2}{А.М. Сахарюк}
\member{Ассистент\\ преподавателя}{М.В. Водяницкий}

% Вывод титульника
\maketitle

% Задание
% Задание
\begin{addition}{Задание}
  Выполнить задания на Python и оформить отчет по стандартам ВВГУ.

  \textit{\textbf{Задание 1.}}  
  Написать программу, которая определяет, как будет вести себя кондиционер. 
  Если температура в помещении 20 градусов и выше, то кондиционер выключается, если меньше - включается. 
  Температура должна вводится пользователем с консол

  \textit{\textbf{Задание 2.}}  
  Год делится на четыре сезона: зима, весна, лето и осень. 
  Написать программу, которая запрашивает у пользователя номер месяца и выводит к какому сезону этот месяц относится.

  \textit{\textbf{Задание 3.}}  
  Считается, что один год, прожитый собакой, эквивалентен семи человеческим годам. 
  При этом зачастую не учитывается, что собаки становятся абсолютно взрослыми уже к двум годам. 
  Таким образом, многие предпочитают каждый из первых двух лет жизни собаки приравнивать к 10.5 годам человеческой жизни, а все последующие к 4.
  Написать программу, которая будет переводить собачий возраст в человеческий.
  Программа должна корректно обрабатывать входные данные и выводить соответствующие
  сообщения об ошибках:
  \begin{vvsu_list}
  \item Если вводится не число
  \item Если вводится число меньше 1 
  \item Если вводится число большее 22
  \end{vvsu_list}
  
  \textit{\textbf{Задание 4.}}  
  Число делиться на 6 только в случае соблюдения двух условий:
  \begin{vvsu_list}
    \item Последняя цифра четная
    \item Сумма всех цифр делиться на 3
  \end{vvsu_list}

  Написать программу, которая выведет делиться ли введенное число на 6 или нет.

  \textit{\textbf{Задание 5.}}  
  Написать программу, которая будет проверять пароль на надежность.
  Пароль считается надежным, если его длина не менее 8 символов и если он содержит:
  \begin{vvsu_list}
   \item Заглавные буквы латиницы
   \item Строчные буквы латиницы
   \item Числа
   \item Специальные знаки
  \end{vvsu_list}
  В случае, если пароль не проходит по одному из условий, необходимо сообщить пользователю каким именно условиям он не удовлетворяет.

  \textit{\textbf{Задание 6.}}  
  Написать программу, которая определяет, является ли введенный пользователем год високосным. 
  Год считается високосным, если он делится на 4, но не делится на 100, либо если он делится на 400.

  \textit{\textbf{Задание 7.}}  
  Написать программу, которая запрашивает у пользователя три числа и выводит на экран наименьшее из них. 
  При решении нельзя использовать встроенные функции min() и max().

  \textit{\textbf{Задание 8.}}  
  В магазине проводится акция. Акция работает по следующим правилам:
  \begin{vvsu_list}
    \item Сумма < 1000 => скидка - 0\%
    \item Сумма < 5000 => скидка - 5\%
    \item Сумма < 10000 => скидка - 10\%
    \item Сумма > 10000 => скидка - 15\%    
  \end{vvsu_list}
  Напишите программу, которая запрашивает сумму покупки и выводит размер скидки и итоговую сумму к оплате.

  \textit{\textbf{Задание 9.}}  
  Написать программу, которая определяет время суток по введенному часу (целое число от 0 до 23).
  \begin{vvsu_list}
    \item С 0 до 5 часов - ночь
    \item С 6 до 11 часов - утро
    \item С 12 до 17 часов - день
    \item  С 18 до 23 часов - вечер
  \end{vvsu_list}


  \textit{\textbf{Задание 10.}}  
  Написать программу, которая определяет, является ли введенное число простым. 
  Число называется простым, если оно больше 1 и делится только на 1 и само себя. 
  Программа должна корректно обрабатывать некорректный ввод и выводить соответствующие сообщения об ошибках.
\end{addition}

% Содержание
\toc

% Глава - Выполнение работы
\section{Выполнение работы}

% Подглава - Задание 1
\subsection{Задание 1}

В данном задании была объявлена переменная отвечающая за ввод пользователя.
Переменная проверяется и в зависимости от выполнения условия выводится сообщение на экран. 
На рисунке \ref{fig:code_task_1} представлен код полученной программы.

\begin{vvsu_figure}{Листинг программы для задания 1}{fig:code_task_1}
  \begin{minipage}{.75\textwidth}
    \lstinputlisting[language=Python,basicstyle=\fontsize{10}{10}\linespread{1}\selectfont\ttfamily]{code/task1.py}
  \end{minipage}
\end{vvsu_figure}

Пояснение работы программы:
\begin{vvsu_list}
  \item Пользователь вводит текущую температуру;
  \item Ввод пользователя сохраняется в переменную user\_input как целое число;
  \item Производится проверка введённого пользователем числа;
  \item Если температура ≥ 20°C - выводится сообщение о выключенном кондиционер;
  \item Eсли температура < 20°C - выводится сообщение о включенном кондиционере.
\end{vvsu_list}

После выполнения программы в консоль выводится включился кондиционер или нет.

% Подглава - Задание 2
\subsection{Задание 2}

В данном задании от пользователя было запрошено число в диапазоне от 1 до 12. 
Ввод пользователя записывается в переменную user\_input. 
Затем значение переменной проверяется и выводится название месяца по его числу. 
В случае выхода из диапазона программа обрабатывает
ошибку.
На рисунке \ref{fig:code_task_2} представлен код полученной программы.

\begin{vvsu_figure}{Листинг программы для задания 2}{fig:code_task_2}
  \begin{minipage}{.75\textwidth}
    \lstinputlisting[language=Python,basicstyle=\fontsize{10}{10}\linespread{1}\selectfont\ttfamily]{code/task2.py}
  \end{minipage}
\end{vvsu_figure}

Пояснение работы программы:
\begin{vvsu_list}
  \item У пользователя запрашивается ввод.
  \item После, ввод пользователя переводится в числовую переменную user\_input через int().
  \item Переменная проходит проверки.
  \item Если user\_input равен 12 или 1 или 2, то выводится «Зима».
  \item Если user\_input равен 3 или 4 или 5, то выводится «Весна».
  \item Если user\_input равен 6 или 7 или 8, то выводится «Лето».
  \item Если user\_input равен 9 или 10 или 11, то выводится «Осень».
  \item Если ни одно из условий не выполнено, то выводится ошибка.
\end{vvsu_list}

В результате выполнения программы, в консоль выводится название месяца по его счету.

% Подглава - Задание 3ыы
\subsection{Задание 3}

В данном задании от пользователя требуется ввести возраст собаки. 
Программа должна принять ввод, обработать на ошибки и вывести значение в человеческих годах на экран.
На рисунке \ref{fig:code_task_3} представлен код полученной программы.

\begin{vvsu_figure}{Листинг программы для задания 3}{fig:code_task_3}
  \begin{minipage}{.75\textwidth}
    \lstinputlisting[language=Python,basicstyle=\fontsize{10}{10}\linespread{1}\selectfont\ttfamily]{code/task3.py}
  \end{minipage}
\end{vvsu_figure}

Пояснение работы программы:
\begin{vvsu_list}
  \item Задаётся функция errors\_check() с аргументом value.
  \item value проверяется на: является ли числом, находится в диапазоне от 1 до 22.
  \item Если одно из условий не удовлетворено, то выводится ошибка и возвращается значение False.
  \item Если все условия удовлетворены, то возвращается значение True.
  \item Задаётся функция calculate\_age() с аргументом value.
  \item Вызывается метод errors\_check() и в него передаётся значение value.
  \item Если возвращается False, то программа заканчивается.
  \item value проверяется.
  \item Если value больше или равно 2, то выводится возраст по формуле $21 + (value – 2) * 4$.
  \item В противном случае выводится 11.5.
  \item У пользователя запрашивается ввод и выводится в dog\_age.
  \item Вызывается метод calculate\_age() и в него передаётся dog\_age.
\end{vvsu_list}

В результате работы программы, в консоль выводится возраст собаки в человеческих годах.

% Подглава - Задание 4
\subsection{Задание 4}

В данном задании от пользователя требуется ввести число. Программа должна вывести делится число на 6 или нет.
На рисунке \ref{fig:code_task_4} представлен код полученной программы.

\begin{vvsu_figure}{Листинг программы для задания 4}{fig:code_task_4}
  \begin{minipage}{.75\textwidth}
    \lstinputlisting[language=Python,basicstyle=\fontsize{10}{10}\linespread{1}\selectfont\ttfamily]{code/task4.py}
  \end{minipage}
\end{vvsu_figure}

Пояснение работы программы:
\begin{vvsu_list}
  \item У пользователя запрашивается число
  \item Ввод пользователя записывается в переменную user\_input.
  \item Объявляется переменная sum\_of\_numbers типа int и равная 0.
  \item sum\_of\_numbers суммируется с каждой цифрой в числе введённом пользователем.
  \item Проверяется условие: если последняя цифра user\_input делится на 2 без остатка, sum\_of\_numbers делится на 3 без остатка и sum\_of\_numbers не равен 0, то выводится в консоль, что число делится на 6.
  \item В противном случае, выводится, что число не делится на 6.  
\end{vvsu_list}

В результате выполнения программы в консоль выводится – делится число на 6 или нет.

% Подглава - Задание 5
\subsection{Задание 5}

В данном задания от пользователя требуется ввести пароль. 
Программа должна проверить пароль по параметрам: не менее 8 символов, есть заглавные, строчные и  специальные символы, есть числа. 
В консоль выводится «надёжный пароль» если все  параметры соблюдены и ошибки, если какие-то параметры не соблюдены.
На рисунке \ref{fig:code_task_5} представлен код полученной программы.

\begin{vvsu_figure}{Листинг программы для задания 5}{fig:code_task_5}
  \begin{minipage}{.75\textwidth}
    \lstinputlisting[language=Python,basicstyle=\fontsize{10}{10}\linespread{1}\selectfont\ttfamily]{code/task5.py}
  \end{minipage}
\end{vvsu_figure}

Пояснение работы программы:
\begin{vvsu_list}
  \item Импортируется класс string.
  \item Объявляется переменная special\_chars и ей присваивается значение равное строке из всех специальных символов.
  \item У пользователя запрашивается ввод и он записывается в переменную user\_input.
  \item Объявляется список для хранения и вывода ошибок.
  \item Пароль проверяется на длину.
  \item Пароль проверяется на наличие заглавных символов.
  \item Пароль проверяется на наличие строчных символов.
  \item Пароль проверяется на наличие специальных символов.
  \item Для каждого невыполненного условия в mistakes добавляется соответствующая строка об ошибке.
  \item В случае, если есть ошибки, то они выводятся в консоль через запятую.
  \item В противном случае выводится «надёжный пароль».
\end{vvsu_list}

В результате выполнения программы в консоль выводится надёжность пароля.

% Подглава - Задание 6
\subsection{Задание 6}

В данном задании требуется пользователю ввести год. 
Программа должна вывести является этот год високосным или нет. 
На рисунке \ref{fig:code_task_6} представлен код программы.

\begin{vvsu_figure}{Листинг программы для задания 6}{fig:code_task_6}
  \begin{minipage}{.75\textwidth}
    \lstinputlisting[language=Python,basicstyle=\fontsize{10}{10}\linespread{1}\selectfont\ttfamily]{code/task6.py}
  \end{minipage}
\end{vvsu_figure}

Пояснение работы программы:
\begin{vvsu_list}
  \item У пользователя запрашивается ввод и он записывается в виде целого числа в переменную user\_input.
  \item Проверяется условие: если user\_input делится на 4 без остатка и user\_input делится на 100 без остатка или user\_input делится на 400 без остатка, то выводится, что год високосный.
  \item В противном случае выводится, что год не високосный.
\end{vvsu_list}

В результате выполнения данной программы в консоль выводится – является ли год високосным, или нет.

% Подглава - Задание 7
\subsection{Задание 7}

В данном задании от пользователя требуется ввести три числа.
Программа должна без функций min() и max() вывести наименьшее число.
На рисунке \ref{fig:code_task_7} представлен код программы.

\begin{vvsu_figure}{Листинг программы для задания 7}{fig:code_task_7}
  \begin{minipage}{.75\textwidth}
    \lstinputlisting[language=Python,basicstyle=\fontsize{10}{10}\linespread{1}\selectfont\ttfamily]{code/task7.py}
  \end{minipage}
\end{vvsu_figure}

Пояснение работы программы:
\begin{vvsu_list}
  \item У пользователя запрашивается ввод и он записывается в лист user\_numbers с разделением через пробел.
  \item Объявляется переменная min\_number и в неё записывается целочисленное первое число из списка user\_numbers.
  \item Если второе число из списка user\_numbers больше min\_number, то оно записывается в min\_number.
  \item Если третье число из списка user\_numbers больше min\_number, то оно записывается в min\_number.
  \item Выводится в консоль значение min\_number.
\end{vvsu_list}

В результате выполнения программы, в консоль выводится наименьшее число из трёх.

% Подглава - Задание 8
\subsection{Задание 8}

В данном задании от пользователя требуется ввести сумму покупки. 
Программа должна рассчитать скидку и вывести сумму со скидкой на экран.
На рисунке \ref{fig:code_task_8} представлен код программы.

\begin{vvsu_figure}{Листинг программы для задания 8}{fig:code_task_8}
  \begin{minipage}{.75\textwidth}
    \lstinputlisting[language=Python,basicstyle=\fontsize{10}{10}\linespread{1}\selectfont\ttfamily]{code/task8.py}
  \end{minipage}
\end{vvsu_figure}

Пояснение работы программы:
\begin{vvsu_list}
  \item У пользователя запрашивается ввод и он записывается в переменную user\_amount
  \item Переменная user\_amount проверяется на значение.
  \item Если user\_amount находится в диапазоне от 0 до 1000, то объявляется переменная discount равная 0.
  \item Если user\_amount находится в диапазоне от 1000 до 5000, то объявляется переменная discount равная 5.
  \item Если user\_amount находится в диапазоне от 5000 до 10000, то объявляется переменная discount равная 10.
  \item В любом другом случае, объявляется переменная discount равная 15.
  \item Объявляется переменная final\_price равная $user\_amount * (100 - discount) / 100$.
  \item В консоль выводятся: сумма покупки, скидка, сумма со скидкой.
\end{vvsu_list}

В результате выполнения данной программы, рассчитывается скидка относительно
стоимости товара и в консоль выводится сумма покупки, скидка, сумма со скидкой.

% Подглава - Задание 9
\subsection{Задание 9}

В данном задании от пользователя требуется ввести некоторый час (число в диапазоне
от 0 до 23). Программе нужно вывести какое сейчас время суток, относительно ввода
пользователя.
На рисунке \ref{fig:code_task_9} представлен код программы.

\begin{vvsu_figure}{Листинг программы для задания 9}{fig:code_task_9}
  \begin{minipage}{.75\textwidth}
    \lstinputlisting[language=Python,basicstyle=\fontsize{10}{10}\linespread{1}\selectfont\ttfamily]{code/task9.py}
  \end{minipage}
\end{vvsu_figure}

Пояснение работы программы:
\begin{vvsu_list}
  \item У пользователя запрашивается ввод и он записывается в переменную user\_time
  \item user\_time проверяется на значение.
  \item Если user\_time больше 23 и меньше 0, то выводится ошибка.
  \item Если user\_time больше 0 и меньше или равно 5, то выводится «Сейчас ночь».
  \item Если user\_time больше 5 и меньше или равно 11, то выводится «Сейчас утро».
  \item Если user\_time больше 11 и меньше 17, то выводится «Сейчас день».
  \item В противном случае выводится «Сейчас вечер».
\end{vvsu_list}

В результате выполнения программы в консоль выводится время суток относительно времени введённого пользователем

% Подглава - Задание 10
\subsection{Задание 10}

В данном задании от пользователя требуется ввести число. 
Программе нужно проверить является ли число простым и вывести это в консоль.
На рисунке \ref{fig:code_task_10} представлен код программы.

\begin{vvsu_figure}{Листинг программы для задания 10}{fig:code_task_10}
  \begin{minipage}{.75\textwidth}
    \lstinputlisting[language=Python,basicstyle=\fontsize{10}{10}\linespread{1}\selectfont\ttfamily]{code/task10.py}
  \end{minipage}
\end{vvsu_figure}

Пояснение работы программы:
\begin{vvsu_list}
  \item Задаётся функция is\_prime() с аргументом n.
  \item is\_prime() возвращает значение False если аргумент n меньше или равен 1.
  \item Объявляется переменная limit равная корню из n и к нему прибавляется 1.
  \item В диапазоне от 2 до limit -1 проверяется делится ли n на какое либо число.
  \item Если число делится, т возвращается False, иначе возвращается True.
  \item У пользователя запрашивается ввод и он записывается в переменную user\_input.
  \item Программа пытается выполнить следующий блок кода, и если сталкивается с ошибкой, то выводит «Ошибка: нужно ввести целое число» в консоль.
  \item Объявляется переменная num равная целому числу user\_input.
  \item Вызывается функция is\_prime() и в неё передаётся переменная num.
  \item Если is\_prime() возвращает True, то выводится, что num – простое число.
  \item В противном случае выводится, что num не является простым числом.
\end{vvsu_list}

В результате выполнения данной программы выводится в консоль – является ли число
введённое пользователем простым или нет.

% \begin{application}{Приложение А}{Приложение А\\ \vspace{1em} Описание процесса работы с драйверами}
%   Самое обычное тестовое приложение
% \end{application}

\end{document}
