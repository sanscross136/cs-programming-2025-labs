\documentclass[]{vvsu}
\usepackage{csquotes}

\vvsuyear{2026}

%%%%%%%%%%%%%%%%%%%

\usepackage{graphicx} % для изображений
\usepackage{tabularray} % для таблиц
\usepackage{siunitx} % для обозначений (процент, градус)
\usepackage{listings} % для листингов кода

% Список путей, где будут искаться изображения и файлы
\graphicspath{{images/}}

% Файл со списком источников (не используется)
% \addbibresource{./references.bib}

% Автор документа
\author{А.М. Сахарюк}

% Настройка стилей для листингов кода
\input{listing_styles.tex}

%%%%%%%%%%%%%%%%%%%

\begin{document}

% Шапка
\vvsuhead{\linespread{1}\selectfont{}МИНОБРНАУКИ РОССИИ\\
\vspace{10pt}Федеральное государственное бюджетное образовательное учреждение\\
высшего образования\\
\fontsize{13}{13}\selectfont{}<<ВЛАДИВОСТОКСКИЙ ГОСУДАРСТВЕННЫЙ УНИВЕРСИТЕТ>>\\
(ФГБОУ ВО <<ВВГУ>>)\\
\vspace{10pt}\fontsize{12}{12}\selectfont{}ИНСТИТУТ ИНФОРМАЦИОННЫХ ТЕХНОЛОГИЙ И АНАЛИЗА ДАННЫХ\\
КАФЕДРА ИНФОРМАЦИОННЫХ ТЕХНОЛОГИЙ И СИСТЕМ}

% Название отчета
\title{Отчет\\по лабораторной работе №7}
\subtitle{по дисциплине\\<<Информатика и программирование>>}

% Участники работы
\member{Студент\\ гр. БИН-25-2}{А.М, Сахарюк}
\member{Ассистент\\ преподавателя}{М.В. Водяницкий}

% Вывод титульника
\maketitle

% Задание
% Задание
\begin{addition}{Задание}
  Выполнить задания на Python и оформить отчет по стандартам ВВГУ.

  \textit{\textbf{Задание 1.}}  


Имеется список объектов Фонда с указанием уровня угрозы:

objects = [
    ("Containment Cell A", 4),
    ("Archive Vault", 1),
    ("Bio Lab Sector", 3),
    ("Observation Wing", 2)
]

Используя sorted и лямбда-выражение, отсортируйте объекты по возрастанию \break уровня угрозы

  \textit{\textbf{Задание 2.}}  
  Дан список сотрудников Фонда с количеством проведенных смен и стоимостью одной смены:

Используя map и лямбда-выражение, создайте список общей стоимости работы каждого сотрудника

Затем найдите максимальную стоимость с помощью max

  \textit{\textbf{Задание 3.}}  
Дан список персонала с уровнем допуска:

personnel = [
    {"name": "Dr. Klein", "clearance": 2},
    {"name": "Agent Brooks", "clearance": 4},
    {"name": "Technician Reed", "clearance": 1}
]

Используя map и лямбда-выражение, создайте новый список, где каждому сотруднику добавляется категория допуска:

    "Restricted" - уровень 1
    "Confidential" - уровни 2–3
    "Top Secret" - уровень 4 и выше

Результат должен быть списком словарей
  
  \textit{\textbf{Задание 4.}}  

Дан список зон Фонда с указанием времени активности (в часах):

Используя filter и лямбда-выражение, выберите зоны, которые полностью работают в дневной период (с 8 до 18 включительно)

  \textit{\textbf{Задание 5.}}  
 Фонд анализирует служебные отчеты. Некоторые отчеты содержат внешние ссылки, которые должны быть удалены перед архивированием

Используя filter и лямбда-выражение:

    Отберите отчеты, содержащие ссылки (http или https)
    Преобразуйте их так, чтобы вместо ссылки отображалось [ДАННЫЕ УДАЛЕНЫ]


  \textit{\textbf{Задание 6.}}  
 Дан список SCP-объектов с указанием их класса содержания:
Используя filter и лямбда-выражение, сформируйте список SCP-объектов, которые требуют усиленных мер содержания

    ⚠️ К объектам с усиленными мерами относятся все SCP, класс которых не равен "Safe"

Результат должен быть списком словарей исходного формата
  \textit{\textbf{Задание 7.}}  
Дан список инцидентов с количеством задействованного персонала:

incidents = [
    {"id": 101, "staff": 4},
    {"id": 102, "staff": 12},
    {"id": 103, "staff": 7},
    {"id": 104, "staff": 20}
]

Используя sorted и лямбда-выражение:

    Отсортируйте инциденты по количеству персонала
    Оставьте только три наиболее ресурсоемких инцидента


  \textit{\textbf{Задание 8.}}  
Дан список протоколов безопасности и их уровней критичности:

protocols = [
    ("Lockdown", 5),
    ("Evacuation", 4),
    ("Data Wipe", 3),
    ("Routine Scan", 1)
]

Используя map и лямбда-выражение, создайте новый список строк вида:

"Protocol Lockdown - Criticality 5"


  \textit{\textbf{Задание 9.}}  
Имеется список смен охраны с указанием длительности (в часах):

shifts = [6, 12, 8, 24, 10, 4]

Используя filter и лямбда-выражение, выберите только те смены, которые:

    длятся не менее 8 часов
    не превышают 12 часов



  \textit{\textbf{Задание 10.}}  
Дан список сотрудников с результатами психологической оценки (от 0 до 100):

evaluations = [
    {"name": "Agent Cole", "score": 78},
    {"name": "Dr. Weiss", "score": 92},
    {"name": "Technician Moore", "score": 61},
    {"name": "Researcher Lin", "score": 88}
]

Используя max и лямбда-выражение, определите сотрудника с наивысшей оценкой

Результатом должно быть имя сотрудника и его балл
\end{addition}

% Содержание
\toc

% Глава - Выполнение работы
\section{Выполнение работы}

% Подглава - Задание 1
\subsection{Задание 1}

В данном задании требуется отсортировать список объектов Фонда по возрастанию уровня угрозы.
На рисунке \ref{fig:code_task_1} представлен код полученной программы.

\begin{vvsu_figure}{Листинг программы для задания 1}{fig:code_task_1}
  \begin{minipage}{.75\textwidth}
    \lstinputlisting[language=Python,basicstyle=\fontsize{10}{10}\linespread{1}\selectfont\ttfamily]{task 1.py}
  \end{minipage}
\end{vvsu_figure}

Пояснение работы программы:
\begin{vvsu_list}
\item Используется встроенная функция sorted
\item В качестве ключа сортировки передаётся лямбда-выражение, которое берёт второй элемент кортежа (item[1])
\item Результат присваивается переменной objects и выводится на экран
\end{vvsu_list}

После выполнения программы в консоль выводится отсортированный список объектов, начиная с наименьшего уровня угрозы.

% Подглава - Задание 2
\subsection{Задание 2}

В данном задании требуется рассчитать общую стоимость работы каждого сотрудника и найти сотрудника с максимальной стоимостью.
На рисунке \ref{fig:code_task_2} представлен код полученной программы.

\begin{vvsu_figure}{Листинг программы для задания 2}{fig:code_task_2}
  \begin{minipage}{.75\textwidth}
    \lstinputlisting[language=Python,basicstyle=\fontsize{10}{10}\linespread{1}\selectfont\ttfamily]{task 2.py}
  \end{minipage}
\end{vvsu_figure}

Пояснение работы программы:
\begin{vvsu_list}
\item Используется функция max с ключом - лямбда-выражением, которое вычисляет произведение shift\_cost на shifts для каждого сотрудника
\item Результат - словарь сотрудника с наибольшей общей стоимостью
\item Выводится имя сотрудника и его максимальная стоимость в форматированной строке
\end{vvsu_list}

После выполнения программы в консоль выводится имя сотрудника и его максимальная общая стоимость работы.

% Подглава - Задание 3ыы
\subsection{Задание 3}

В данном задании требуется добавить каждому сотруднику категорию доступа на основе его уровня.
На рисунке \ref{fig:code_task_3} представлен код полученной программы.

\begin{vvsu_figure}{Листинг программы для задания 3}{fig:code_task_3}
  \begin{minipage}{.75\textwidth}
    \lstinputlisting[language=Python,basicstyle=\fontsize{10}{10}\linespread{1}\selectfont\ttfamily]{task 3.py}
  \end{minipage}
\end{vvsu_figure}

Пояснение работы программы:
\begin{vvsu_list}
\item Используется map и лямбда-выражение для перебора списка сотрудников
\item К каждому словарю сотрудника добавляется новое поле "category", значение которого берётся из словаря AccessLevels по ключу clearance
\item Результат преобразуется в список и выводится на экран
\end{vvsu_list}
После выполнения программы в консоль выводится обновлённый список сотрудников, каждый из которых содержит поле "category" с соответствующей категорией доступа.

% Подглава - Задание 4
\subsection{Задание 4}

В данном задании требуется отфильтровать зоны, которые работают строго в дневное время (с 8 до 18 часов включительно).
На рисунке \ref{fig:code_task_4} представлен код полученной программы.

\begin{vvsu_figure}{Листинг программы для задания 4}{fig:code_task_4}
  \begin{minipage}{.75\textwidth}
    \lstinputlisting[language=Python,basicstyle=\fontsize{10}{10}\linespread{1}\selectfont\ttfamily]{task 4.py}
  \end{minipage}
\end{vvsu_figure}

Пояснение работы программы:
\begin{vvsu_list}
\item Используется функция filter с лямбда-выражением
\item Условие: active\_from ≤ 8 и active\_to ≥ 18 — то есть зона активна всё время с 8 до 18
\item Результат преобразуется в список и выводится на экран
\end{vvsu_list}

После выполнения программы в консоль выводится список зон, которые полностью покрывают дневной период с 8 до 18 часов.

% Подглава - Задание 5
\subsection{Задание 5}

В данном задании требуется отфильтровать отчёты, содержащие внешние ссылки, и заменить их на текст [данные удалены].
На рисунке \ref{fig:code_task_5} представлен код полученной программы.

\begin{vvsu_figure}{Листинг программы для задания 5}{fig:code_task_5}
  \begin{minipage}{.75\textwidth}
    \lstinputlisting[language=Python,basicstyle=\fontsize{10}{10}\linespread{1}\selectfont\ttfamily]{task 5.py}
  \end{minipage}
\end{vvsu_figure}

Пояснение работы программы:
\begin{vvsu_list}
\item Используется filter с лямбда-выражением, проверяющим наличие "http" или "https" в поле "text"
\item Отобранные отчёты преобразуются: в поле "text" ссылки заменяются на строку "[данные удалены]" с помощью map и лямбды
\item Результат выводится как список обновлённых отчётов
\end{vvsu_list}

После выполнения программы в консоль выводится список отчётов, в которых все внешние ссылки заменены на [данные удалены].

% Подглава - Задание 6
\subsection{Задание 6}

В данном задании требуется отобрать SCP-объекты, которые требуют усиленных мер содержания - то есть все объекты, чей класс не равен "Safe".
На рисунке \ref{fig:code_task_6} представлен код программы.

\begin{vvsu_figure}{Листинг программы для задания 6}{fig:code_task_6}
  \begin{minipage}{.75\textwidth}
    \lstinputlisting[language=Python,basicstyle=\fontsize{10}{10}\linespread{1}\selectfont\ttfamily]{task 6.py}
  \end{minipage}
\end{vvsu_figure}

Пояснение работы программы:
\begin{vvsu_list}
\item Используется filter с лямбда-выражением, проверяющим условие: scp\_object['class'] != 'Safe'
\item Результат преобразуется в список и выводится на экран
\end{vvsu_list}

После выполнения программы в консоль выводится список словарей - всех SCP-объектов, кроме тех, что имеют класс "Safe".

% Подглава - Задание 7
\subsection{Задание 7}

В данном задании требуется отсортировать инциденты по убыванию количества персонала и оставить только три самых ресурсоёмких.
На рисунке \ref{fig:code_task_7} представлен код программы.

\begin{vvsu_figure}{Листинг программы для задания 7}{fig:code_task_7}
  \begin{minipage}{.75\textwidth}
    \lstinputlisting[language=Python,basicstyle=\fontsize{10}{10}\linespread{1}\selectfont\ttfamily]{task 7.py}
  \end{minipage}
\end{vvsu_figure}

Пояснение работы программы:
\begin{vvsu_list}
\item Используется sorted с лямбда-выражением, сортирующим по значению "staff" в порядке убывания (reverse=True)
\item Из отсортированного списка берутся первые три элемента с помощью среза [:3]
\item Результат выводится на экран
\end{vvsu_list}

После выполнения программы в консоль выводится список из трёх инцидентов с наибольшим количеством задействованного персонала.

% Подглава - Задание 8
\subsection{Задание 8}

В данном задании требуется преобразовать список кортежей с названиями протоколов и их критичностью в список строк в заданном формате.
На рисунке \ref{fig:code_task_8} представлен код программы.

\begin{vvsu_figure}{Листинг программы для задания 8}{fig:code_task_8}
  \begin{minipage}{.75\textwidth}
    \lstinputlisting[language=Python,basicstyle=\fontsize{10}{10}\linespread{1}\selectfont\ttfamily]{task 8.py}
  \end{minipage}
\end{vvsu_figure}

Пояснение работы программы:
\begin{vvsu_list}
\item Используется map и лямбда-выражение, которое для каждого кортежа формирует строку: "Protocol {название} - Criticality {критичность}"
\item Результат преобразуется в список и выводится на экран
\end{vvsu_list}

После выполнения программы в консоль выводится список строк, каждая из которых содержит название протокола и его уровень критичности в указанном формате.

% Подглава - Задание 9
\subsection{Задание 9}

В данном задании требуется отфильтровать смены охраны, длительность которых находится в диапазоне от 8 до 12 часов включительно.
На рисунке \ref{fig:code_task_9} представлен код программы.

\begin{vvsu_figure}{Листинг программы для задания 9}{fig:code_task_9}
  \begin{minipage}{.75\textwidth}
    \lstinputlisting[language=Python,basicstyle=\fontsize{10}{10}\linespread{1}\selectfont\ttfamily]{task 9.py}
  \end{minipage}
\end{vvsu_figure}

Пояснение работы программы:
\begin{vvsu_list}
\item Используется filter с лямбда-выражением, проверяющим условие: shift >= 8 and shift <= 12
\item Результат преобразуется в список и выводится на экран
\end{vvsu_list}
После выполнения программы в консоль выводится список смен, длина которых составляет от 8 до 12 часов включительно.

% Подглава - Задание 10
\subsection{Задание 10}
В данном задании требуется найти сотрудника с наивысшей оценкой из списка.
На рисунке \ref{fig:code_task_10} представлен код программы.

\begin{vvsu_figure}{Листинг программы для задания 10}{fig:code_task_10}
  \begin{minipage}{.75\textwidth}
    \lstinputlisting[language=Python,basicstyle=\fontsize{10}{10}\linespread{1}\selectfont\ttfamily]{task 10.py}
  \end{minipage}
\end{vvsu_figure}

Пояснение работы программы:
\begin{vvsu_list}
\item Используется функция max с лямбда-выражением, выбирающим элемент с максимальным значением по ключу "score"
\item Выводится имя и балл этого сотрудника в форматированной строке
\end{vvsu_list}

После выполнения программы в консоль выводится имя сотрудника и его максимальный балл.

% \begin{application}{Приложение А}{Приложение А\\ \vspace{1em} Описание процесса работы с драйверами}
%   Самое обычное тестовое приложение
% \end{application}

\end{document}
