\documentclass[]{vvsu}
\usepackage{csquotes}

\vvsuyear{2026}

%%%%%%%%%%%%%%%%%%%

\usepackage{graphicx} % для изображений
\usepackage{tabularray} % для таблиц
\usepackage{siunitx} % для обозначений (процент, градус)
\usepackage{listings} % для листингов кода

% Список путей, где будут искаться изображения и файлы
\graphicspath{{images/}}

% Файл со списком источников (не используется)
% \addbibresource{./references.bib}

% Автор документа
\author{А.М. Сахарюк}

% Настройка стилей для листингов кода
\input{listing_styles.tex}

%%%%%%%%%%%%%%%%%%%

\begin{document}

% Шапка
\vvsuhead{\linespread{1}\selectfont{}МИНОБРНАУКИ РОССИИ\\
\vspace{10pt}Федеральное государственное бюджетное образовательное учреждение\\
высшего образования\\
\fontsize{13}{13}\selectfont{}<<ВЛАДИВОСТОКСКИЙ ГОСУДАРСТВЕННЫЙ УНИВЕРСИТЕТ>>\\
(ФГБОУ ВО <<ВВГУ>>)\\
\vspace{10pt}\fontsize{12}{12}\selectfont{}ИНСТИТУТ ИНФОРМАЦИОННЫХ ТЕХНОЛОГИЙ И АНАЛИЗА ДАННЫХ\\
КАФЕДРА ИНФОРМАЦИОННЫХ ТЕХНОЛОГИЙ И СИСТЕМ}

% Название отчета
\title{Отчет\\по лабораторной работе №6}
\subtitle{по дисциплине\\<<Информатика и программирование>>}

% Участники работы
\member{Студент\\ гр. БИН-25-2}{А.М. Сахарюк}
\member{Ассистент\\ преподавателя}{М.В. Водяницкий}

% Вывод титульника
\maketitle

% Задание
% Задание
\begin{addition}{Задание}
  Выполнить задания на Python и оформить отчет по стандартам ВВГУ.

  \textit{\textbf{Задание 1.}}  
  Написать функцию, которая конвертирует время из одной \break величины в другую.
  \begin{vvsu_list}
  \item На вход подается:
  \item число (величина времени)
  \item исходная единица измерения
  \item единица измерения, в которую нужно перевести
  \end{vvsu_list}
  Функция должна вернуть конвертированное значение

  \textit{\textbf{Задание 2.}}  
  Пользователь делает вклад в банке в размере a рублей сроком на n лет

  Процент по вкладу зависит от суммы и срока

  Зависимость от суммы:
  \begin{vvsu_list}
    \item каждые 10 000 рублей увеличивают ставку на 0.3\%
    \item но суммарное увеличение не может превышать 5\%
    \item инимальный вклад - 30 000 рублей
  \end{vvsu_list}
  Зависимость от срока:
  \begin{vvsu_list}
    \item первые 3 года - 3\%
    \item от 4 до 6 лет - 5\%
   \item  более 6 лет - 2\%
  \end{vvsu_list}
Необходимо написать функцию, которая рассчитывает прибыль пользователя без учета первоначально вложенной суммы

Используется сложный процент: каждый год процент начисляется на текущую сумму вклада

На вход подаются: сумма вклада и количество лет. Результат: сумма прибыли (не весь вклад, а только заработанные проценты)

  \textit{\textbf{Задание 3.}}  
  Написать функцию для вывода всех простых чисел в заданном диапазоне. Нужно учитывать некорректные данные (например, начало больше конца или диапазон без простых чисел)

  На вход подаются два числа: начало и конец диапазона (включительно). На выходе - список всех простых чисел или сообщение об ошибке
  
  \textit{\textbf{Задание 4.}} 
  Реализовать функцию сложения двух матриц

При сложении двух матриц получается новая матрица того же размера, где каждый элемент - это сумма элементов с тем же индексом из двух исходных матриц

Ограничения:
  \begin{vvsu_list}
    \item складывать можно только матрицы одинакового размера
    \item размер матрицы должен быть строго больше 2 (например, 3×3, 4×4 и т.д.)
    \item  при нарушении условий нужно вывести сообщение об ошибке
  \end{vvsu_list}
На вход подаются:
   \begin{vvsu_list}
   \item  размер матрицы n (для квадратной матрицы n × n)
   \item  элементы первой матрицы (по строкам, через пробел)
   \item  элементы второй матрицы в таком же формате
  \end{vvsu_list}
Результат - новая матрица (в том же формате), либо сообщение об ошибке
  \textit{\textbf{Задание 5.}}  
  Написать функцию, которая определяет, является ли строка палиндромом

  Палиндром - это строка, которая читается одинаково слева направо и справа налево (обычно без учета пробелов, регистра и знаков препинания - эти правила нужно явно задать в своей реализации)

  На вход подается строка. На выходе:
   \begin{vvsu_list}
   \item  Да, если это палиндром
   \item  Нет, если это не палиндром
  \end{vvsu_list}

\end{addition}

% Содержание
\toc

% Глава - Выполнение работы
\section{Выполнение работы}

% Подглава - Задание 1
\subsection{Задание 1}

В данном задании требуется написать функцию для конвертирования времени.
На рисунке \ref{fig:code_task_1} представлен код полученной программы.

\begin{vvsu_figure}{Листинг программы для задания 1}{fig:code_task_1}
  \begin{minipage}{.75\textwidth}
    \lstinputlisting[language=Python,basicstyle=\fontsize{10}{10}\linespread{1}\selectfont\ttfamily]{task1.py}
  \end{minipage}
\end{vvsu_figure}

Пояснение работы программы:
\begin{vvsu_list}
  \item Объявляется словарь TimeUnits с соотношениями единиц времени в секундах
  \item Определяется функция ConvertTime, принимающая время и целевую единицу
  \item В функции из входной строки убирается единица, число переводится в секунды
  \item Результат делится на значение целевой единицы и возвращается с \break её обозначением
  \item Программа запрашивает у пользователя ввод (время и целевая единица)
  \item Вызывает функцию и выводит результат на экран
\end{vvsu_list}


После выполнения программы в консоль выводится сконвертированное значение времени в указанной единице.
% Подглава - Задание 2

\subsection{Задание 2}

В данном задании требуется написать функцию для расчёта прибыли по банковскому вкладу с учётом сложных процентов и зависимой ставки.
На рисунке \ref{fig:code_task_2} представлен код полученной программы.

\begin{vvsu_figure}{Листинг программы для задания 2}{fig:code_task_2}
  \begin{minipage}{.75\textwidth}
    \lstinputlisting[language=Python,basicstyle=\fontsize{10}{10}\linespread{1}\selectfont\ttfamily]{task2.py}
  \end{minipage}
\end{vvsu_figure}

\begin{vvsu_list}
  \item Проверяется, что сумма вклада не меньше 30 000 рублей
  \item Определяется базовая ставка в зависимости от срока: 3\% — до 3 лет, 5\% — 4–6 лет, 2\% — более 6 лет
  \item Рассчитывается дополнительная ставка: 0.3\% за каждые 10 000 рублей, но не более 5\% суммарно
  \item Итоговая ставка — минимум из (базовая + доп.) и 5\%
  \item Прибыль рассчитывается по формуле сложного процента: вклад\break \* (1 \+ ставка)\^лет \- вклад
  \item Результат округляется до двух знаков и возвращается
\end{vvsu_list}

После выполнения программы в консоль выводится сумма заработанных процентов (прибыль) за указанный срок.

% Подглава - Задание 3ыы
\subsection{Задание 3}

В данном задании требуется написать функцию для вывода всех простых чисел в заданном диапазоне с обработкой некорректных данных.
На рисунке \ref{fig:code_task_3} представлен код полученной программы.

\begin{vvsu_figure}{Листинг программы для задания 3}{fig:code_task_3}
  \begin{minipage}{.75\textwidth}
    \lstinputlisting[language=Python,basicstyle=\fontsize{10}{10}\linespread{1}\selectfont\ttfamily]{task3.py}
  \end{minipage}
\end{vvsu_figure}

Пояснение работы программы:
\begin{vvsu_list}
  \item Определяется функция IsPrime, проверяющая, является ли число простым (до корня из числа)
  \item Определяется функция PrintAllPrimes, принимающая начало и конец диапазона
  \item Проверяются входные данные: начало ≥ 0 и начало ≤ конец — иначе\break выводится "Error!"
  \item Перебираются все числа в диапазоне, для каждого вызывается IsPrime
  \item Простые числа добавляются в строку через пробел
  \item Если ни одного простого числа не найдено — выводится "Error!", иначе — строка с числами
\end{vvsu_list}

После выполнения программы в консоль выводится строка с простыми числами в указанном диапазоне или сообщение "Error!" при некорректном вводе или отсутствии простых чисел.

% Подглава - Задание 4
\subsection{Задание 4}

В данном задании требуется написать функцию для сложения двух квадратных матриц одинакового размера.
На рисунке \ref{fig:code_task_4} представлен код полученной программы.

\begin{vvsu_figure}{Листинг программы для задания 4}{fig:code_task_4}
  \begin{minipage}{.75\textwidth}
    \lstinputlisting[language=Python,basicstyle=\fontsize{10}{10}\linespread{1}\selectfont\ttfamily]{task4.py}
  \end{minipage}
\end{vvsu_figure}

Пояснение работы программы:
\begin{vvsu_list}
  \item Проверяется, что размер матрицы n строго больше 2 — иначе \break возвращается "Error!"
  \item Создаётся пустой результатирующий список Result
  \item Внешний цикл по строкам, внутренний — по столбцам: складываются элементы с одинаковыми индексами из обеих матриц
  \item Суммы добавляются в строки результата, строки — в общую матрицу
  \item Функция возвращает итоговую матрицу или "Error!" при нарушении условий
  \item Программа считывает размер и две матрицы, вызывает функцию и выводит результат построчно
\end{vvsu_list}

После выполнения программы в консоль выводится результирующая матрица (по строкам через пробел) или сообщение "Error!" при неверном размере или формате.

% Подглава - Задание 5
\subsection{Задание 5}

В данном задании требуется написать функцию для проверки, является ли строка палиндромом (без учёта пробелов, регистра и знаков препинания).
На рисунке \ref{fig:code_task_5} представлен код полученной программы.

\begin{vvsu_figure}{Листинг программы для задания 5}{fig:code_task_5}
  \begin{minipage}{.75\textwidth}
    \lstinputlisting[language=Python,basicstyle=\fontsize{10}{10}\linespread{1}\selectfont\ttfamily]{task5.py}
  \end{minipage}
\end{vvsu_figure}

Пояснение работы программы:
\begin{vvsu_list}
  \item Импортируется модуль string для работы со знаками препинания
  \item Строка приводится к нижнему регистру
  \item Удаляются все пробелы и знаки препинания из строки
  \item Сравнивается полученная строка с её обратной версией
  \item Если совпадает — возвращается "Да", иначе — "Нет"
\end{vvsu_list}

После выполнения программы в консоль выводится "Да", если строка — палиндром, или "Нет" — если нет.

% \begin{application}{Приложение А}{Приложение А\\ \vspace{1em} Описание процесса работы с драйверами}
%   Самое обычное тестовое приложение
% \end{application}

\end{document}
