\begin{introduction}

Разработка консольной текстовой RPG представляет собой важный этап в освоении основ программирования. Этот проект объединяет в себе работу с объектно-ориентированными принципами, алгоритмами, обработкой пользовательского ввода и управлением состоянием игры --- всё это без необходимости использования графических библиотек.

Целью данной работы является создание прототипа текстовой RPG, соответствующего техническому заданию, с акцентом на читаемость кода, логическую структуру и корректную реализацию игровых механик.

{Гейм-дизайн}

При проектировании игры были определены следующие ключевые\\ принципы гейм-дизайна:

\begin{vvsu_list}
  \item Простота управления: игра полностью управляется через текстовое меню с числовым вводом, что обеспечивает доступность и предсказуемость для пользователя.
  \item Прогрессия персонажа: игрок ощущает рост силы через повышение уровня, распределение очков характеристик и получение нового снаряжения.
  \item Случайность и выбор: каждый запуск игры уникален благодаря процедурной генерации комнат, врагов и предметов; при этом игрок сохраняет контроль над стратегией --- куда идти, кого атаковать, что экипировать.
  \item Баланс классов: три класса (Воин, Лучник, Маг) имеют разные стартовые параметры и стратегии развития, что поощряет повторные прохождения и эксперименты.
\end{vvsu_list}

Эти принципы легли в основу архитектурных и программных решений, принятых при реализации прототипа.
\end{introduction}